\documentclass{report}
\usepackage[utf8]{inputenc}
\usepackage[russian]{babel}
\usepackage{setspace,amsmath}
\usepackage{amssymb}
\usepackage{amsthm}
\usepackage{amsfonts}
\usepackage{scalerel}
\usepackage{graphicx}
\usepackage{float}
\usepackage{wrapfig}
\usepackage[unicode, pdftex]{hyperref}

\def\stretchint#1{\vcenter{\hbox{\stretchto[440]{\displaystyle\int}{#1}}}}
\def\scaleint#1{\vcenter{\hbox{\scaleto[3ex]{\displaystyle\int}{#1}}}}

\theoremstyle{definition}
\newtheorem{definition}{Определение}[section]
\newtheorem{example}{Пример}
\newtheorem*{effect}{Следствие}
\newtheorem{statement}{Утверждение}[section]
\newtheorem*{remark}{Замечание}
\newtheorem{lemma}{Лемма}[section]
\newtheorem{theorem}{Теорема}[section]

\title{Дискретная математика и математическая логика \\ 3 семестр}
\author{Данил Заблоцкий}
\date{\today}

\begin{document}

\maketitle
\tableofcontents
\chapter*{Введение}

В прошлом году изучались:
\begin{enumerate}
    \item Основы
          \begin{itemize}
              \item Булевы функции
              \item Формулы логики высказываний
              \item Эквивалентные преобразования
              \item Нормальные формы
                    \begin{itemize}
                        \item ДНФ/КНФ
                        \item СДНФ/СКНФ
                        \item Полином Жегалкина
                    \end{itemize}
              \item Минимальная ДНФ
          \end{itemize}
    \item Теория булевых функций
          \begin{itemize}
              \item Основной объект - булевы функции
              \item Суперпозиция и подстановка переменных
              \item Замкнутые классы булевых функций
              \item Замыкание
              \item Полные системы булевых функций и базисы
              \item Теорема Поста о полноте системы булевых функций
                    \begin{itemize}
                        \item Классы Поста
                        \item Леммы о немонотонных, несамодвойственных, нелинейных функциях
                        \item Теорема
                        \item Теорема о максимально замкнутых классах
                    \end{itemize}
          \end{itemize}
    \item Логика высказываний
          \begin{itemize}
              \item Основной объект - формулы
              \item Основы теории доказательств
                    \begin{itemize}
                        \item Логическое следование
                        \item Вывод в форматных системах
                        \item Исчисление высказываний
                        \item Теорема Геделя о полноте исчисления высказываний
                        \item Исчисление высказываний Генцена
                        \item Метод резолции для логики высказываний
                    \end{itemize}
          \end{itemize}
\end{enumerate}

В этом году будет изучаться \textbf{язык логики предикатов}.

Пример Аристотеля: $\left\{
    \begin{array}{l}
        $Все люди - смертные$ \\
        $Сократ - человек$
    \end{array}
    \right. \implies$ Сократ - смертный

$x$: Все люди - смертные \\

$y$: Сократ - человек \\

$z$: Сократ - смертный

\begin{center}
    {\Large $x, y \nvDash z$}
\end{center}

Вывод: ЛВ обладает слабой выразительной силой по сравнению с естественным языком.

\chapter{Основные понятия}

\begin{definition}
    \textbf{$n$-местный предикант} на множестве $A$ - это отображение вида:
    \begin{center}
        $P: \ A^n \rightarrow \{0,1\}$,
    \end{center} при этом $n$-местность - $P$.
\end{definition}

Формально, предикант - это высказывание, зависящее от параметров.

\begin{example}
    \begin{enumerate}
        \item $A = \mathbb{Z}$.

              $P(x) = 1 \iff x$ - простое число.

              $Q(x,y) = 1 \iff x + y = 1$

              $R(x,y) = 1 \iff x < y$

              $T(x,y,z) = 1 \iff z =$ НОД$(x,y)$

        \item $A$ - множество людей.

              Примеры предикатов на $A$:

              $P(x) = 1 \iff x$ - женщина

              $Q(x,y) = 1 \iff x$ - родитель $y$

              $R(x,y) = 1 \iff x$ и $y$ - братья
    \end{enumerate}
\end{example}

\begin{definition}
    $n$-местная операция на множестве $A$ - это отображение вида $f: \ A^n \rightarrow A$.
\end{definition}

\begin{example}
    $A = \mathbb{Z}$.
    \begin{enumerate}
        \item $f_1(x) = x + 1$;
        \item $f_2(x) = 2x$;
        \item $f_3(x) = 0$;
        \item $f_4(x) = x^2$;
        \item $g_1(x,y) = \left\{
            \begin{array}{ll}
                x^y, & y > 0; \\
                0, & y \leqslant $ иначе$
            \end{array}
            \right.$;
            \item $g_2(x,y) = x + y$;
            \item $g_3(x,y) =$ сумма последних цифр чисел $x$ и $y$.
    \end{enumerate}

    \begin{equation*}
        \forall x (P(x) \& Q(x) \rightarrow R(f(x)))
    \end{equation*}
\end{example}

\begin{remark}
    Чтобы писать формулы, достаточно иметь только обозначение предикатов и операций и знать их местность.
\end{remark}

\begin{definition}
    \textbf{Сигнатура} - набор трех непересакающихся множеств: $\mathfrak{R} \cup \mathfrak{F} \cup \mathfrak{C}$, где элементы множества $\mathfrak{R}$ назовем предикатные символы, элементы $\mathfrak{F}$ - функциональные символы, элементы $\mathfrak{C}$ - константные символы. Так же должна быть определена функция $\mathfrak{M}: \ \mathfrak{R} \cup \mathfrak{F} \rightarrow \mathbb{N}$ - местность символов. 

    \textbf{Сигнатура} - это набор предикатных, функциональных и константных символов с указанием их местности.
\end{definition}

\begin{example}
    $\Sigma = \{P^{(1)}, \ Q^{(2)}, \ f^{(1)}, \ g^{(2)}, \ c\}$, где $P^{(1)}, \ Q^{(2)}$ - предикат, $f^{(1)}, \ g^{(2)}$ - функциональные символы, а $c$ - константный.
\end{example}

Символы $P, Q, R, \ldots$ считаем предикатными, символы $f, g, h, \ldots$ - функциональными, $a, b, c, \ldots$ - константами.

Сигнатуры $\Sigma$ и $\Gamma$ - равны, если есть содержание и одинаковые количества символов каждого сорта, и местности символов попарно равны.
\begin{center}
    $\Sigma = \{P^{(1)}, \ Q^{(2)}, \ f^{(1)}, \ g^{(2)}, \ a, \ b\} =$ \\

    $\Gamma = \{P_1^{(1)}, \ P_2^{(2)}, \ f_1^{(1)}, \ f_2^{(2)}, \ c_1, \ c_2\}$
\end{center}

Иногда элементы сигнатуры представляются общепринятыми символами ($+, \ \cdot, \ \ldots$).

\begin{example}
    Имеем формулу: $\forall x \ (P(x) \rightarrow Q(f(x)))$. Эта формула истинна или ложна?

    Для ответа не хватает:
    \begin{itemize}
        \item множества, из которого берутся значения переменных;
        \item расшифровки того, что обозначают символы $P, \ Q, \ f$.
    \end{itemize}
\end{example}

\begin{definition}
    \textbf{Интепретация сигнатуры} $\Sigma$ на множестве $A$ - это отображение $I$, которое:
    \begin{itemize}
        \item каждый предикатный символ $P^{(n)} \in \Sigma$ отображает в $n$-местный предикат на множестве $A$;
        \item каждый функциональный символ $f^{(n)} \subset \Sigma$ отображает в $n$-местную операцию на $A$;
        \item каждый константный символ отображает в элемент множество $A$.
    \end{itemize}
\end{definition}

\begin{definition}
    \textbf{Алгебраическая система} - это набор, состаящий из множества $A$, сигнатуры $\Sigma$ и интепретации сигнатуры $\Sigma$ на множестве 
\end{definition}

\end{document}