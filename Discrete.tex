\documentclass{report}
\usepackage[utf8]{inputenc}
\usepackage[russian]{babel}
\usepackage{setspace,amsmath}
\usepackage{amssymb}
\usepackage{amsthm}
\usepackage{amsfonts}
\usepackage{scalerel}
\usepackage{graphicx}
\usepackage{float}
\usepackage{wrapfig}
\usepackage[unicode, pdftex]{hyperref}

\def\stretchint#1{\vcenter{\hbox{\stretchto[440]{\displaystyle\int}{#1}}}}
\def\scaleint#1{\vcenter{\hbox{\scaleto[3ex]{\displaystyle\int}{#1}}}}

\theoremstyle{definition}
\newtheorem{definition}{Определение}[section]
\newtheorem{example}{Пример}
\newtheorem*{effect}{Следствие}
\newtheorem{statement}{Утверждение}[section]
\newtheorem*{remark}{Замечание}
\newtheorem{lemma}{Лемма}[section]
\newtheorem{theorem}{Теорема}[section]

\title{Дискретная математика и математическая логика \\ 3 семестр}
\author{Данил Заблоцкий}
\date{\today}

\begin{document}

\maketitle
\tableofcontents
\chapter*{Введение}

В прошлом году изучались:
\begin{enumerate}
    \item Основы
          \begin{itemize}
              \item Булевы функции
              \item Формулы логики высказываний
              \item Эквивалентные преобразования
              \item Нормальные формы
                    \begin{itemize}
                        \item ДНФ/КНФ
                        \item СДНФ/СКНФ
                        \item Полином Жегалкина
                    \end{itemize}
              \item Минимальная ДНФ
          \end{itemize}
    \item Теория булевых функций
          \begin{itemize}
              \item Основной объект -- булевы функции
              \item Суперпозиция и подстановка переменных
              \item Замкнутые классы булевых функций
              \item Замыкание
              \item Полные системы булевых функций и базисы
              \item Теорема Поста о полноте системы булевых функций
                    \begin{itemize}
                        \item Классы Поста
                        \item Леммы о немонотонных, несамодвойственных, нелинейных функциях
                        \item Теорема
                        \item Теорема о максимально замкнутых классах
                    \end{itemize}
          \end{itemize}
    \item Логика высказываний
          \begin{itemize}
              \item Основной объект -- формулы
              \item Основы теории доказательств
                    \begin{itemize}
                        \item Логическое следование
                        \item Вывод в форматных системах
                        \item Исчисление высказываний
                        \item Теорема Геделя о полноте исчисления высказываний
                        \item Исчисление высказываний Генцена
                        \item Метод резолции для логики высказываний
                    \end{itemize}
          \end{itemize}
\end{enumerate}

В этом году будет изучаться \textbf{язык логики предикатов}.

Пример Аристотеля: $\left\{
    \begin{array}{l}
        $Все люди -- смертные$ \\
        $Сократ -- человек$
    \end{array}
    \right. \implies$ Сократ -- смертный

$x$: Все люди -- смертные \\

$y$: Сократ -- человек \\

$z$: Сократ -- смертный

\begin{center}
    {\Large $x, y \nvDash z$}
\end{center}

Вывод: ЛВ обладает слабой выразительной силой по сравнению с естественным языком.

\chapter{Основные понятия}

\begin{definition}[$n$-местный предикант]
    \textbf{$n$-местный предикант} на множестве $A$ -- это отображение вида:
    \begin{center}
        $P: \ A^n \rightarrow \{0,1\}$,
    \end{center} при этом $n$-местность -- $P$.
\end{definition}

Формально, предикант -- это высказывание, зависящее от параметров.

\begin{example}
    \begin{enumerate}
        \item $A = \mathbb{Z}$.

              $P(x) = 1 \iff x$ -- простое число.

              $Q(x,y) = 1 \iff x + y = 1$

              $R(x,y) = 1 \iff x < y$

              $T(x,y,z) = 1 \iff z =$ НОД$(x,y)$

        \item $A$ -- множество людей.

              Примеры предикатов на $A$:

              $P(x) = 1 \iff x$ -- женщина

              $Q(x,y) = 1 \iff x$ -- родитель $y$

              $R(x,y) = 1 \iff x$ и $y$ -- братья
    \end{enumerate}
\end{example}

\begin{definition}[$n$-местная операция]
    \textbf{$n$-местная операция} на множестве $A$ -- это отображение вида $f: \ A^n \rightarrow A$.
\end{definition}

\begin{example}
    $A = \mathbb{Z}$.
    \begin{enumerate}
        \item $f_1(x) = x + 1$;
        \item $f_2(x) = 2x$;
        \item $f_3(x) = 0$;
        \item $f_4(x) = x^2$;
        \item $g_1(x,y) = \left\{
                  \begin{array}{ll}
                      x^y, & y > 0;               \\
                      0,   & y \leqslant $ иначе$
                  \end{array}
                  \right.$;
        \item $g_2(x,y) = x + y$;
        \item $g_3(x,y) =$ сумма последних цифр чисел $x$ и $y$.
    \end{enumerate}

    \begin{equation*}
        \forall x (P(x) \& Q(x) \rightarrow R(f(x)))
    \end{equation*}
\end{example}

\begin{remark}
    Чтобы писать формулы, достаточно иметь только обозначение предикатов и операций и знать их местность.
\end{remark}

\begin{definition}[Сигнатура]
    \textbf{Сигнатура} -- набор трех непересакающихся множеств: $\mathfrak{R} \cup \mathfrak{F} \cup \mathfrak{C}$, где элементы множества $\mathfrak{R}$ назовем предикатные символы, элементы $\mathfrak{F}$ - функциональные символы, элементы $\mathfrak{C}$ - константные символы. Так же должна быть определена функция $\mathfrak{M}: \ \mathfrak{R} \cup \mathfrak{F} \rightarrow \mathbb{N}$ - местность символов.

    \textbf{Сигнатура} -- это набор предикатных, функциональных и константных символов с указанием их местности.
\end{definition}

\begin{example}
    $\Sigma = \{P^{(1)}, \ Q^{(2)}, \ f^{(1)}, \ g^{(2)}, \ c\}$, где $P^{(1)}, \ Q^{(2)}$ -- предикат, $f^{(1)}, \ g^{(2)}$ -- функциональные символы, а $c$ -- константный.
\end{example}

Символы $P, Q, R, \ldots$ считаем предикатными, символы $f, g, h, \ldots$ -- функциональными, $a, b, c, \ldots$ -- константами.

Сигнатуры $\Sigma$ и $\Gamma$ -- равны, если есть содержание и одинаковые количества символов каждого сорта, и местности символов попарно равны.
\begin{center}
    $\Sigma = \{P^{(1)}, \ Q^{(2)}, \ f^{(1)}, \ g^{(2)}, \ a, \ b\} =$ \\

    $\Gamma = \{P_1^{(1)}, \ P_2^{(2)}, \ f_1^{(1)}, \ f_2^{(2)}, \ c_1, \ c_2\}$
\end{center}

Иногда элементы сигнатуры представляются общепринятыми символами ($+, \ \cdot, \ \ldots$).

\begin{example}
    Имеем формулу: $\forall x \ (P(x) \rightarrow Q(f(x)))$. Эта формула истинна или ложна?

    Для ответа не хватает:
    \begin{itemize}
        \item множества, из которого берутся значения переменных;
        \item расшифровки того, что обозначают символы $P, \ Q, \ f$.
    \end{itemize}
\end{example}

\begin{definition}[Интерпретация сигнатуры]
    \textbf{Интепретация сигнатуры} $\Sigma$ на множестве $A$ -- это отображение $I$, которое:
    \begin{itemize}
        \item каждый предикатный символ $P^{(n)} \in \Sigma$ отображает в $n$-местный предикат на множестве $A$;
        \item каждый функциональный символ $f^{(n)} \subset \Sigma$ отображает в $n$-местную операцию на $A$;
        \item каждый константный символ отображает в элемент множество $A$.
    \end{itemize}
\end{definition}

\begin{definition}[Алгебраическая система]
    \textbf{Алгебраическая система} -- это набор, состоящий из множества $A$, сигнатуры $\Sigma$ и интепретации сигнатуры $\Sigma$ на множестве $A$. Множество $A$ называется \textbf{основным множеством системы}.
    \begin{equation*}
        \phi = <A, \Sigma, \mathfrak{I}>
    \end{equation*}
\end{definition}

\section{Язык логики предикатов (1-го порядка)}

Пусть $\Sigma$ -- сигнатура. Алфавит языка логики предикатов сигнатуры $\Sigma$ -- это
\begin{equation*}
    A_\Sigma = \Sigma \cup \{x_1, x_2, \ldots, \land, \lor, \lnot, \rightarrow, \iff, (, ), \forall, \exists, ,\}
\end{equation*}

\begin{definition}[Терм]
    \textbf{Терм} сигнатуры $\Sigma$ -- это слово, построенное по правилам:
    \begin{enumerate}
        \item Символ переменной -- терм.
        \item Константный символ -- терм.
        \item Если $f^{(n)} \in \Sigma$ -- функциональный символ и $t_1,\ldots,t_n$ -- термы, то слово $f(t_1,\ldots,t_n)$ -- тоже терм.
    \end{enumerate}
\end{definition}

\begin{example}
    $\Sigma = \{+^{(2)},\cdot^{(2)},<^{(2)},0,1\}$

    Термы: $x,y,z,0,1,x+y,z\cdot x, 0 < 1, x\cdot 1, z(x+y), x\cdot 1 + y,\ldots$
\end{example}

\begin{definition}[Атомарные формулы]
    \textbf{Атомарные формулы} сигнатуры $\Sigma$ -- слово одного из двух видов:
    \begin{itemize}
        \item $t_1 = t_2$, где $t_1,t_2$ -- термы;
        \item $P(t_1,t_2)$, где $P^{(n)} \in \Sigma$ -- предикатный символ, $t_1,\ldots,t_n$ -- термы.
    \end{itemize}
\end{definition}

\begin{definition}[Формула языка логики предикатов]
    \textbf{Формула языка логики предикатов} -- это слово, построенное по правилам:
    \begin{enumerate}
        \item Атомарная формула -- формула.
        \item Если $\phi_1$ и $\phi_2$ -- формулы, то слова $(\phi_1 \land \phi_2), \ (\phi_1 \lor \phi_2), \ (\phi_1 \rightarrow \phi_2), \ (\phi_1 \iff \phi_2), \ \lnot\phi_1$ -- тоже формулы.
        \item Если $\phi$ -- формула, $x$ -- переменная, то $(\forall x \ \phi)$ и $(\exists x \ \phi)$ -- тоже формулы.
    \end{enumerate}
\end{definition}

\begin{example}
    $x = y \land y = 1, \ x < y \land y < z \lor z = 0, \ x + y < 1 \lor x + y < 0 \lor x = y, \ \forall x,y \ (x < y \lor y < x \lor x \leqslant y)$.
\end{example}

Для уменьшения количества скобок в формулах используется соглашение о приоритетах:
\begin{center}
    $(\forall,\exists) \rightarrow \lnot \rightarrow \land \rightarrow$ остальные операции
\end{center}

\begin{definition}[Связное и свободное вхождение]
    Вхождение переменной $x$ в формуле вида $(\forall x \ \phi)$ и $(\exists x \ \phi)$ назовем \textbf{связным}. В противном случае вхождение переменной \textbf{свободное}.

    Переменная $x$ \textbf{свободная} в формуле $\phi$, если есть хотя бы одно ее свободное вхождение в $\phi$. Иначе переменная \textbf{связная}.

    Запись $\phi(x_1,\ldots,x_n)$ означает, что все свободные переменные формулы и содержатся в строке $x_1,\ldots,x_n$.
\end{definition}

\begin{definition}[Замкнутая формула (предложение)]
    \textbf{Замкнутая формула (предложение)} -- это формула без свободных переменных.
\end{definition}

\begin{example}
    $\forall x \exists y \quad (Q(x,y) \lor \lnot P(x))$
\end{example}

Язык = синтаксис + семантика

Синтаксис -- правила написания слов

Семантика -- смысл слов

\section{Семантика языка логики предикатов}

Пусть $A = <A,\Sigma>$ -- алгебраическая система. Терм $t(x_1,\ldots,x_n)$ определяет в системе $A$ функцию $t_A: \ A^n\rightarrow A$ по следующему правилу: заменяя в терме $t$ все функции, константные знаки на все интепретации, получаем суперпозицию функций.

\begin{example}
    $\Sigma = \{P^{(1)},Q^{(2)}, f^{(1)}, g^{(2)},a,b\}$
    \begin{equation*}
        t_1(x,y) = g(f(x), y), \quad t_2(x,y) = f(g(a,g(b,f(x))))
    \end{equation*}
\end{example}

\begin{definition}
    Пусть $\phi(x_1,\ldots,x_n)$ -- формула, $A = <A,\Sigma>$ -- алгебраическая система, $a_1,\ldots,a_n \in A$. Истинность формулы $\phi$ в алгебраической системе $A$ на элементах $a_1,\ldots,a_n \ (A \vDash \phi(a_1,\ldots,a_n))$ определяется по следующим правилам:
    \begin{enumerate}
        \item Пусть $\phi$ имеет вид $t_1(x_1,\ldots,x_n) = t_2(x_1,\ldots,x_n)$, где $t_1$ и $t_2$ -- формулы:
              \begin{equation*}
                  A \vDash \phi(a_1,\ldots,a_n) \iff t_{1_A}(a_1,\ldots,a_n) = t_{2_A}(a_1,\ldots,a_n).
              \end{equation*}
        \item Пусть $\phi$ имеет вид $P(t_1,\ldots, t_k)$, где $P^{(k)}\in \Sigma$ -- предикатный символ, $t_1(x_1,\ldots,x_k),\ldots,t_k(x_1,\ldots,x_n)$ -- термы.
              \begin{equation*}
                  A\vDash \phi(a_1,\ldots,a_n) \iff P_A(t_1(x_1,\ldots,x_n),\ldots,t_k(x_1,\ldots,x_n)) = 1
              \end{equation*}
        \item Пусть $\phi = (\phi_1 \land \phi_2) \ [(\phi_1 \lor \phi_2), \ (\phi_1 \rightarrow \phi_2), \ (\phi_1 \iff \phi_2), \ \lnot \phi_1]$. Истинность определяется из значений формул по таблицам истинности логических связок.
        \item Пусть $\phi(x_1,\ldots,x_n)$ имеет вид $\exists x \ \phi(x,x_1,\ldots,x_n)$. Тогда $A\vDash \phi(a_1,\ldots,a_n) \iff$ для некоторого $b\in A \quad A\vDash \phi(b,a_1,\ldots,a_n)$.
        \item Пусть $\phi(x_1,\ldots,x_n)$ имеет вид $\forall x \ \phi(x,x_1,\ldots,x_n)$. Тогда $A\vDash\phi(a_1,\ldots,a_n) \iff$ для всех $b\in A \quad A\vDash\phi(b,a_1,\ldots,a_n)$.
    \end{enumerate}
\end{definition}

\begin{definition}[Тождественно истинная (ложная), выполнимая формула]
    Формула $\phi(x_1,\ldots,x_n)$ \textbf{тождественно истинна (ложна)} в системе $A$, если $a_1,\ldots,a_n \in A, \ A \vDash \phi(a_1,\ldots,a_n) \ (A\nvDash \phi(a_1,\ldots,a_n))$.

    Формула $\phi$ тождественно \textbf{истинна (ложна)}, если $\phi$ тождественно истинна (ложна) в любой системе сигнатуры $\Sigma$. \\

    Формула $\phi(x_1,\ldots,x_n)$ \textbf{выполнима} в системе $A$, если $\exists a_1,\ldots,a_n: \ A\vDash\phi(a_1,\ldots,a_n)$.

    Формула $\phi$ \textbf{выполнима}, если она выполнима хотя бы в одной системе сигнатуры $\Sigma$.
\end{definition}


\end{document}